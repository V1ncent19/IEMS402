\documentclass[11pt,a4paper]{ctexart}
%以下为所使用的宏包
\usepackage{ulem}%下划线
\usepackage{amsmath,amsfonts,amssymb,amsthm,amsbsy}%数学符号
\usepackage{graphicx}%插入图片
\usepackage{booktabs}%三线表
%\usepackage{indentfirst}%首行缩进
\usepackage{tikz}%作图
\usepackage{appendix}%附录
\usepackage{array}%多行公式/数组
\usepackage{makecell}%表格缩并
\usepackage{siunitx}%SI单位--\SI{number}{unit}
\usepackage{mathrsfs}%数学字体
\usepackage{enumitem}%列表间距
\usepackage{multirow}%列表横向合并单元格
\usepackage[colorlinks,linkcolor=red,anchorcolor=blue,citecolor=green]{hyperref}%超链接引用
\usepackage{float}%图片、表格位置排版
\usepackage{pict2e,keyval,fp,diagbox}%带有斜线的表格
\usepackage{fancyvrb,listings}%设置代码插入环境
\usepackage{minted}%代码环境设置
\usepackage{fontspec}%字体设置
\usepackage{color,xcolor}%颜色设置
\usepackage{titlesec} %自定义标题格式
\usepackage{tabularx}%列表扩展
\usepackage{authblk}%titlepage作者信息
\usepackage{nicematrix}%更好的矩阵标定
\usepackage{fbox}%更多浮动体盒子



%以下是页边距设置
\usepackage[left=0.5in,right=0.5in,top=0.81in,bottom=0.8in]{geometry}

%以下是段行设置
\linespread{1.4}%行距
\setlength{\parskip}{0.1\baselineskip}%段距
\setlength{\parindent}{2em}%缩进


%其他设置
\numberwithin{equation}{section}%公式按照章节编号
\newenvironment{point}{\raggedright$\blacktriangleright$}{}
\newenvironment{algorithm}[1]{\vspace{12pt} \hrule\hrule \vspace{3pt} \noindent\textbf{\color[HTML]{E63F00}Algorithm } \,\textit{#1} \vspace{3pt} \hrule\vspace{6pt}}{\vspace{6pt}\hrule\hrule \vspace{12pt}} % 算法伪代码格式环境


%代码环境\lst设置
\definecolor{CodeBlue}{HTML}{268BD2}
\definecolor{CodeBlue2}{HTML}{0000CD}
\definecolor{CodeGreen}{HTML}{2AA1A2}
\definecolor{CodeRed}{HTML}{CB4B16}
\definecolor{CodeYellow}{HTML}{B58900}
\definecolor{CodePurPle}{HTML}{D33682}
\definecolor{CodeGreen2}{HTML}{859900}
\lstset{
    basicstyle=\tt,%字体设置
    numbers=left, %设置行号位置
    numberstyle=\tiny\color{black}, %设置行号大小
    keywordstyle=\color{black}, %设置关键字颜色
    stringstyle=\color{CodeRed}, %设置字符串颜色
    commentstyle=\color{CodeGreen}, %设置注释颜色
    frame=single, %设置边框格式
    escapeinside=`, %逃逸字符(1左面的键),用于显示中文
    %breaklines, %自动折行
    extendedchars=false, %解决代码跨页时,章节标题,页眉等汉字不显示的问题
    xleftmargin=2em,xrightmargin=2em, aboveskip=1em, %设置边距
    tabsize=4, %设置tab空格数
    showspaces=false, %不显示空格
    emph={TRUE,FALSE,NULL,NAN,NA,<-,},emphstyle=\color{CodeBlue2}, %其他高亮}
}


%节标题格式设置
\titleformat{\section}[block]{\large\bfseries}{Exercise \arabic{section}}{1em}{}[]
\titleformat{\subsection}[block]{}{    \arabic{section}.(\alph{subsection})}{1em}{}[]
% \titleformat{\subsubsection}[block]{\normalsize\bfseries}{    \arabic{subsection}-\alph{subsubsection}}{1em}{}[]
% \titleformat{\paragraph}[block]{\small\bfseries}{[\arabic{paragraph}]}{1em}{}[]


% \titleformat{\sectioncommand}[shape]{format}{title-label}{sep}{before-title}[after-title]



% 中文字号
% 初号42pt, 小初36pt, 一号26pt, 小一24pt, 二号22pt, 小二18pt, 三号16pt, 小三15pt, 四号14pt, 小四12pt, 五号10.5pt, 小五9pt


\begin{document}

\begin{center}\thispagestyle{plain}

{\LARGE\textbf{IEMS 402 Statistical Learning - 2025 Winter}}

{\Large\textbf{HW8}}

Tuorui Peng\footnote{TuoruiPeng2028@u.northwestern.edu}
\end{center}

\thispagestyle{myheadings}\markright{Compiled using \LaTeX}
\pagestyle{myheadings}\markright{Tuorui Peng}




\section{Hilbert Embedding of Probability}



\subsection{}

Consider the functional $ L: f\mapsto \mathbb{E}\left[ f(X) \right]  $ which is a bounded linear functional. By Riesz Representation Theorem, there exists a unique $ h_L\in \mathcal{H} $ such that 
\begin{align*}
     \forall f\in\mathcal{H}:\, \left\langle h_L,f \right\rangle = L(f)=\mathbb{E}_P\left[ f(X) \right]=& \mathbb{E}_P\left[\left\langle \varphi (X), f \right\rangle  \right]  =\left\langle \mathbb{E}_P\left[ \varphi (X) \right] ,f  \right\rangle 
\end{align*}
i.e. such $ \mathcal{H}\ni h_L = \mathbb{E}_P\left[ \varphi (X) \right]  $.


\subsection{}


We prove the contrapositive. Suppose $ \mathbb{E}_P\left[ \varphi (X) \right] = \mathbb{E}_Q\left[ \varphi(X) \right] $, then we consider the following setting: $ \forall \varepsilon >0 $, $ \forall f\in \mathcal{X} $, $ \exists h_{f,\varepsilon }\in \mathcal{H} $ s.t. $ \left\Vert f-h_{f,\varepsilon } \right\Vert _\infty <\varepsilon  $, and we have
\begin{align*}
    \mathbb{E}_P\left[ h_{f,\varepsilon }(X) \right]=& \mathbb{E}_P\left[ \left\langle h_{f,\varepsilon }, \varphi (X) \right\rangle   \right]\\
    =& \left\langle \mathbb{E}_P\left[ \varphi (X) \right] ,h_{f,\varepsilon }\right\rangle \\
    \mathbb{E}_Q\left[ h_{f,\varepsilon }(X) \right]=& \mathbb{E}_Q\left[ \left\langle h_{f,\varepsilon }, \varphi (X) \right\rangle   \right]\\
    =& \left\langle \mathbb{E}_Q\left[ \varphi (X) \right] ,h_{f,\varepsilon }\right\rangle
\end{align*}
So we have
\begin{align*}
    \left\vert \mathbb{E}_P\left[ f(X) \right] - \mathbb{E}_Q\left[ f(X) \right]   \right\vert \leq & 2\varepsilon + \left\vert \mathbb{E}_P\left[ h_{f,\varepsilon }(X) \right] - \mathbb{E}_Q\left[ h_{f,\varepsilon }(X) \right]   \right\vert \\
    =& 2\varepsilon + \left\vert \left\langle \mathbb{E}_P\left[ \varphi (X) \right] - \mathbb{E}_Q\left[ \varphi (X) \right] ,h_{f,\varepsilon } \right\rangle  \right\vert \\
    =& 2\varepsilon 
\end{align*}
Note that the above statement is true $ \forall \varepsilon >0,\, \forall f\in \mathcal{C} $, thus proves the contrapositive that $ P\mathop{ = }\limits^{\mathrm{ d } } Q $ must hold. And we have if $ P\mathop{ \neq }\limits^{\mathrm{ d } } Q $, then $ \mathbb{E}_P\left[ \varphi (X) \right] \neq \mathbb{E}_Q\left[ \varphi (X) \right] $.


\subsection{}

We have for right hand side:
\begin{align*}
     \mathrm{R.H.S.}=&\sqrt{\mathbb{E}\left[ k(X,X') \right] + \mathbb{E}\left[ k(Z,Z') \right]  -2\mathbb{E}\left[ k(X,Z) \right] }\\
     =& \sqrt{ \left\langle \varphi (X), \varphi (X') \right\rangle + \left\langle \varphi (Z), \varphi (Z') \right\rangle -2\left\langle \varphi (X), \varphi (Z) \right\rangle }\\
     =& \sqrt{\left\langle \varphi (X)-\varphi (X') , \varphi (Z)-\varphi (Z') \right\rangle } \\
    :=& E
\end{align*}

For left hand side:
\begin{align*}
    \mathop{ \sup }\limits_{f\in\mathcal{H},\left\Vert f \right\Vert \leq 1}\left\vert \mathbb{E}_P\left[ f(X) \right] -\mathbb{E}_Q\left[ f(X) \right] \right\vert =& \mathop{ \sup }\limits_{f\in\mathcal{H},\left\Vert f \right\Vert \leq 1}\left\vert \left\langle \left\langle \mathbb{E}\left[ \varphi (X)-\varphi (Z) \right] ,f  \right\rangle  \right\rangle  \right\vert 
\end{align*}
which reach maximum when $ f\propto \mathbb{E}\left[ \varphi (X)-\varphi (Z) \right] := \alpha \mathbb{E}\left[ \varphi (X)-\varphi (Z) \right] $, in which $ \alpha  $ is taken to make $ \left\Vert f \right\Vert =1 $. Thus we have
\begin{align*}
    1=\alpha ^2\left\langle \mathbb{E}\left[ \varphi (X)-\varphi (Z) \right],\mathbb{E}\left[ \varphi (X)-\varphi (Z) \right] \right\rangle =& \alpha ^2E^2  \Rightarrow \alpha =\frac{1}{E} 
\end{align*}

Substitute back to the left hand side, we have
\begin{align*}
     \mathrm{L.H.S.}=&\mathop{ \sup }\limits_{f\in\mathcal{H},\left\Vert f \right\Vert \leq 1}\left\vert \mathbb{E}_P\left[ f(X) \right] -\mathbb{E}_Q\left[ f(X) \right] \right\vert \\
     =& \left\vert \left\langle \mathbb{E}\left[ \varphi (X)-\varphi (Z) \right] ,\frac{1}{E}\mathbb{E}\left[ \varphi (X)-\varphi (Z) \right]  \right\rangle  \right\vert \\
    =& \frac{1}{E}\left\langle \mathbb{E}\left[ \varphi (X)-\varphi (Z) \right] ,\mathbb{E}\left[ \varphi (X)-\varphi (Z) \right]  \right\rangle \\
    =& \frac{1}{E}E=E = \mathrm{R.H.S.}
\end{align*}


\section{Example of Kernel}

\subsection{}
We can verify the condition for $ k_\mathrm{ norm }  $ to be a kernel function by checking the positive semi-definiteness and symmetry easily:
\begin{align*}
    k_\mathrm{ norm }  (x,z) = & k_\mathrm{ norm } (z,x)\\
    \forall x_1^m,\,\alpha _1^n,\, \sum_{i,j=1}^n k_\mathrm{ norm }(x_i,x_j)\alpha _i\alpha _j = & \sum_{i,j=1}^n k(x_i,z_i)\dfrac{ \alpha _i }{ \sqrt{k}(x_i,x_i) }\dfrac{ \alpha _j }{ \sqrt{k(x_j,x_j)} }\geq 0  
\end{align*}

\subsection{}

We prove the reproducing property by checking the following:
\begin{align*}
    \forall f,\,\forall x\,:\,\left\langle k(x,\, \cdot \, ), f \right\rangle  =& \int_0^1 k'(x,z)f'(z)\,\mathrm{d}z\\
    =& \int_0^1 \mathbf{1}_{[0,x]}(z)\mathbf{1}_{[0,x]}(z)f'(z)\,\mathrm{d}z\\
    =& \int_0^x f'(z)\,\mathrm{d}z\\
    =& f(x)
\end{align*}

And we can easily verify the symmetry and positive semi-definiteness of $ k(\, \cdot \, ,\, \cdot \, ) = \, \cdot \, \wedge \, \cdot \,  $ as follows:
\begin{align*}
    k(x,z) = & x\wedge z = z\wedge x = k(z,x)\\
    \forall g:\, \int_0^1\int_0^1 g(x)g(z)k(x,z)\,\mathrm{d}x\,\mathrm{d}z = & \int_0^1\int_0^1 g(x)g(z)\left\langle \mathbf{1}_{[0,x]},\mathbf{1}_{[0,z]} \right\rangle \,\mathrm{d}x\,\mathrm{d}z\\
    =& \int_0^1\int_0^1 \left\langle g(x)\mathbf{1}_{[0,x]}, g(z)\mathbf{1}_{[0,z]}  \right\rangle \,\mathrm{d}x\,\mathrm{d}z\\
    =& \left\Vert \int_0^1 g(x)\mathbf{1}_{[0,x]} \,\mathrm{d}x \right\Vert ^2\geq 0.
\end{align*}


\subsection{}

WLOG take $ f^{(i)}(0)=0 $ for all $ i\leq k-1 $. So that $ \left\langle f,g \right\rangle = \int _0^1 f^{(k)}(x)g^{(k)}(x) \,\mathrm{d}x  $.

Using similar integration by parts idea, we should have: for each given $ x $, the function $ k(x,\, \cdot \, ) $ satisfies
\begin{align*}
    g(x)=& \left\langle k(x,\, \cdot \, ), g \right\rangle \\
    =& \int_0^1 k^{(k)}(x,z)g^{(k)}(z)\,\mathrm{d}z
\end{align*}
By the Taylor expansion of $ g $ at $ 0 $ with integraion remainders, i.e.
\begin{align*}
     g(x) =& \int_0^x \dfrac{ g^{(k)}(z) }{ (k-1)! } (x-z)^{k-1}\,\mathrm{d}z 
\end{align*}
we have
\begin{align*}
    k^{(k)}(x,z) = \dfrac{ (x-z)^{k-1} }{ (k-1)! } \mathbf{1}_{[0,x]}(z) = \dfrac{ (x-z)^{k-1}_+ }{ (k-1)! }
\end{align*}
and thus
\begin{align*}
    k(x,z) =& \left\langle  k(x,\, \cdot \, ), k(z,\, \cdot \, ) \right\rangle \\
    =& \int_0^1 k^{(k)}(x,u)k^{(k)}(z,u)\,\mathrm{d}u\\
    =& \int_0^1 \dfrac{ (x-u)^{k-1}_+ }{ (k-1)! }\dfrac{ (z-u)^{k-1}_+ }{ (k-1)! }\,\mathrm{d}u
\end{align*}


\section{$ \varphi  $-divergence DRO and Variance Regularization}


Optimzation problem is formalized as:
\begin{align*}
    \mathop{ \sup }\limits_{P\in\mathcal{P}_n }\mathbb{E}_P\left[ \ell(\theta ,X) \right] ,\quad s.t.\, D_\varphi (P\Vert \hat{P}_n) \leq\dfrac{ \rho  }{ n } 
\end{align*}
Since the empirical distribution $ \hat{P}_n $ is a Dirac measure, the optimizer would also be a Dirac measure supported on $ \mathrm{ supp }(X_1^n)  $. We denoted this PMF as:
\begin{align*}
    P:\, X=X_i,\quad w.p. \, p_i 
\end{align*}
So the optimization problem can be reformulated as:
\begin{align*}
    \mathop{ \sup }\limits_{p_i\geq 0,\,\sum_{i=1}^n p_i=1}\sum_{i=1}^n p_i\ell(\theta ,X_i) ,\quad s.t.\, D_\varphi (P\Vert \hat{P}_n) \leq\dfrac{ \rho  }{ n } 
\end{align*}
Lagrangian:
\begin{align*}
    \mathcal{L}(\vec{p};\lambda ,\mu )=& \sum_{i=1}^n p_i\ell(\theta ,X_i) + \lambda \left( \sum_{i=1}^n np_i^2 - 1 - \dfrac{ 2\rho  }{ n }  \right) + \mu \left( \sum_{i=1}^n p_i - 1 \right),\quad \lambda \leq 0
\end{align*}
which is maximized w.r.t. $ \vec{p} $ when 
\begin{align*}
    p_i^* = - \dfrac{ \ell(X_i,\theta )+\mu  }{ 2\lambda n }
\end{align*}
and gives dual problem:
\begin{align*}
     \theta _D(\lambda ,\mu )=& -\sum_{i=1}^n \dfrac{ (\ell(X_i,\theta )+\mu )^2 }{ 4\lambda n }-\lambda (1+\dfrac{ 2\rho  }{ n } ) -\mu ,\quad \lambda \geq 0 
\end{align*}
Which is minimized w.r.t. $ \lambda ,\mu  $:
\begin{align*}
    0=\begin{cases}
        \dfrac{\partial^{}  }{\partial \lambda ^{} }\theta _D =& \sum_{i=1}^n \dfrac{ (\ell(X_i,\theta )+\mu )^2 }{ 4\lambda ^2n }- (1+\dfrac{ 2\rho  }{ n } ) \\
    \dfrac{\partial^{}  }{\partial \mu ^{} }\theta _D =& -\sum_{i=1}^n \dfrac{ (\ell(X_i,\theta )+\mu ) }{ 2\lambda n }-1
    \end{cases} \Rightarrow \begin{cases}
        \mu =\sqrt{\dfrac{ var_{\hat{P}_n}(\ell(X,\theta )) }{ 2\rho /n } } - \mathbb{E}_{\hat{P}_n}\left[ \ell(X,\theta ) \right] \\
        \lambda =-\dfrac{ 1 }{ 2 } \sqrt{\dfrac{ var_{\hat{P}_n}(\ell(X,\theta )) }{ 2\rho /n } }
    \end{cases}
\end{align*}
and gives (with strong duality):
\begin{align*}
    R_n(\theta ,\mathcal{P}_n)=&  \mathop{ \sup }\limits_{P\in\mathcal{P}_n }\mathbb{E}_P\left[ \ell(\theta ,X) \right] = \mathop{ \inf }\limits_{\lambda \leq 0,\mu }\theta _D(\lambda ,\mu ) \\
    =& \mathbb{E}_{\hat{P}_n}\left[ \ell(X,\theta ) \right] + \sqrt{\dfrac{2\rho \mathrm{var}_{\hat{P}_n}(\ell(X,\theta )) }{  n } }
\end{align*}

And we revisit the solution $ p_i^* $:
\begin{align*}
    p_i^* = - \dfrac{ \ell(X_i,\theta )+\mu  }{ 2\lambda n } = & \dfrac{ 1 }{ n }+ \dfrac{ \sqrt{2\rho }(\ell(X_i,\theta )- \mathbb{E}_{\hat{P}_n}\left[ \ell(X,\theta ) \right] ) }{ n\sqrt{n} \sqrt{\mathrm{var}_{\hat{P}_n}(\ell(X,\theta ))} }  
\end{align*}
could satisfy $ p_i^*\geq 0 $ if the empirical variance is large enough \& $ \rho  $ is chosen small enough.


\section{Derive the dual formulation of the Sinkhorn distance}

\subsection{}
The Lagrangian is given by:
\begin{align*}
    \mathcal{L}(\gamma ; u,v)=& \left\langle \gamma ,C \right\rangle -\varepsilon  H(\gamma ) + u'(\gamma \mathbf{1}-a) + v'(\gamma ^T\mathbf{1}-b) 
\end{align*}
miimizing w.r.t. $ \gamma  $ gives the optimality condition: 
\begin{align*}
    \dfrac{\partial^{} \mathcal{L} }{\partial \gamma_{ij} }=&  C_{ij} + \varepsilon (\log \gamma_{ij} + 1) + u_i + v_j = 0 \Rightarrow \gamma_{ij} = \exp\left( -\dfrac{ C_{ij} + u_i + v_j }{ \varepsilon }  - 1 \right)
\end{align*}
And we have dual problem:
\begin{align*}
    \mathop{ \inf }\limits_{u,v}\mathcal{L}(\gamma ; u,v) = & \mathop{ \inf }\limits_{u,v}\left\langle \gamma ,C \right\rangle -\varepsilon  H(\gamma ) + u'(\gamma \mathbf{1}-a) + v'(\gamma ^T\mathbf{1}-b)\\
    =& \mathop{ \inf }\limits_{u,v}\quad -u'a- v'b - \varepsilon \sum_{i,j}\exp\left( -\dfrac{ C_{ij} + u_i + v_j }{ \varepsilon }  - 1 \right)
\end{align*}
with re-parametrization $ u\mapsto -u - \varepsilon ,\quad v\mapsto -v $ and omitting constant, we have dual problem:
\begin{align*}
    \mathop{ \inf }\limits_{u,v}\quad  u'a+v'b  - \varepsilon \sum_{i,j}\exp\left( \dfrac{   u_i + v_j -C_{ij}}{ \varepsilon }   \right)
\end{align*}

\subsection{}

Once the optimal $ u^* $ is known, it's left to solve for $ v^* $:
\begin{align*}
     &\mathop{ \inf }\limits_{v}\quad  v'b  - \varepsilon \sum_{i,j}\exp\left( \dfrac{   u_i^* + v_j -C_{ij}}{ \varepsilon }   \right)v \\
      \Rightarrow& v_j^* = \dfrac{ \varepsilon  }{ b_j } \sum_{i}\exp\left( \dfrac{   u_i^* + v_j^* -C_{ij}}{ \varepsilon }   \right)  \Rightarrow v^* ,\quad \forall j
\end{align*}
which can be solved by fixed point iteration.






































    





















\end{document}
